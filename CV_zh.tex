%% 唯一需要修改的地方是 \photo[64pt]{face} 把face换成自己的照片名
%% start of file `template_en.tex'.
%% Copyright 2006-1008 Xavier Danaux (xdanaux@gmail.com).
%
% This work may be distributed and/or modified under the
% conditions of the LaTeX Project Public License version 1.3c,
% available at http://www.latex-project.org/lppl/.


\documentclass[11pt,a4paper]{moderncv}

% moderncv themes
\moderncvtheme[blue]{classic}                 % optional argument are 'blue' (default), 'orange', 'red', 'green', 'grey' and 'roman' (for roman fonts, instead of sans serif fonts)
%\moderncvtheme[green]{classic}                % idem

% adjust the page margins
\usepackage[scale=0.8]{geometry}
\usepackage[noindent]{ctex}
\setlength{\hintscolumnwidth}{2.5cm}    % 设置日期栏宽度

%\setlength{\hintscolumnwidth}{3cm} % if you want to change the width of the column with the dates
%\AtBeginDocument{\setlength{\maketitlenamewidth}{6cm}}  % only for the classic theme, if you want to change the width of your name placeholder (to leave more space for your address details
\AtBeginDocument{\recomputelengths }                     % required when changes are made to page layout lengths
% personal data
\firstname{胡XX}
\familyname{}
\title{简历}               % optional, remove the line if not wanted
\address{北京市海淀区中科院计算所}{}   % optional, remove the line if not wanted
\mobile{(+086)155XXXXXXXX} {}                   % optional, remove the line if not wanted
%\homepage{www.phodal.com}                  % 可选项、如不需要可删除本行 
%\photo[64pt][0.2pt]{picture}                  % ‘64pt’是图片必须压缩至的高度、‘0.4pt‘是图片边框的宽度 (如不需要可调节至0pt)、’picture‘ 是图片文件的名字;可选项、如不需要可删除本行
%\phone{010-62286960}                      % optional, remove the line if not wanted
%\fax{fax (optional)}                          % optional, remove the line if not wanted
\email{ustc.sosohu@gmail.com}                      % optional, remove the line if not wanted
%\extrainfo{Already recommended for admission to be a postgraduate} % optional, remove the line if not wanted name of the picture file; optional, remove the line if not wanted
%\quote{Job intention: Intern}                 % optional, remove the line if not wanted
\nopagenumbers{}                             % uncomment to suppress automatic page numbering for CVs longer than one page


%----------------------------------------------------------------------------------
%            content
%----------------------------------------------------------------------------------
\begin{document}
\maketitle

\section{教育背景}

\cventry{2013.9--现在}{硕士}{中科院计算所}{}{}
{\textbf{主要课程: }   计算机体系结构, 云计算, 信息检索, TCP/IP 协议等}  % arguments 3 to 6 are optional

\cventry{2009.9--2013.6}{本科}{中国科学技术大学}{\quad}{GPA: 3.56/4.3,\quad  RANK: 17/106}
{\textbf{主要课程: }  数据结构, 算法导论, 操作系统, 计算机网络, 数据库, 编译原理等}  % arguments 3 to 6 are optional


\section{相关经历}
\subsection{实习经历}
\cventry{2014.1--2014.7}{Intel公司}{}{北京}{}
{在Intel公司SSG部门实习半年,主要从事 Intel Xeon Phi众核平台上的高性能并行应用程序性能优化和分析开源MapReduce架构(phoenix++),熟练了使用Intel平台上程序优化相关工具,熟悉了高性能优化的各种手段,加深对Mapreduce架构的理解\\}% arguments 3 to 6 are optional
\cventry{2013.6--2013.9}{中科龙芯公司}{}{合肥}{}
{在中科龙芯系统软件部门实习三个月,主要从事chromium浏览器的移植和优化的工作,整个工作环境都是在龙芯3A机器上进行,通过实习,锻炼了程序性能分析、查询性能瓶颈和解决瓶颈的能力\\}% arguments 3 to 6 are optional
\subsection{项目经历}
\cventry{2013.1--2013.5}{矩阵SVD算法的并行与优化}{}{}{}
{在四核的MIPS架构机器平台上,实现了双精度浮点类型的一般矩阵奇异值分解的并行化,得到了比较理想的加速比,并对最经典的SVD单边Jacobi分解算法作出一些独创性的改进,提高了算法的效率\\}                % arguments 3 to 6 are optional
\cventry{2012.3--2012.5}{C0语言(C语言子集)编译器实现}{}{}{}
{使用flex+bisson的组合,配合自己定义的C0语言文法,建造C0语言AST树,然后产生中间代码,再被解释器转化为汇编代码,最终形成简易编译器\\}                % arguments 3 to 6 are optional

\section{计算机技能}
\cvline{Unix/Linux:}{熟悉Unix/Linux环境开发,熟悉shell、awk等脚本工具}
\cvline{编程语言:}{熟悉C/C++编程}
\cvline{高性能计算:}{熟悉OpenMP, MPI并行编程}
\cvline{其他:}{熟悉hadoop,分布式计算的基本知识}

\section{获得奖项}
\cvline{2012.9}{中国科大计算机学院8211专项奖学金}
\cvline{2011.10}{中国科大RoboGame机器人大赛季军(top 5\%)}
\cvline{2009--2011}{中国科大优秀学生奖学金铜奖}

\section{语言能力}
\cvlanguage{英语}{CET6}{}

\section{兴趣爱好}
\cvline{运动}{ 足球,篮球}
\cvline{读书}{ 计算机经典书籍,各大圣经}
\cvline{旅游}{ 到处旅游,放松身心}

\end{document}


%% end of file `template_en.tex'.
